\documentclass{article}
\usepackage[utf8]{inputenc}
\usepackage{dtklogos} % for BibTeX stylized logo

\title{ESTIMATING TO A GOOD APPROXIMATION THE LIKELYHOOD OF RAIN IN MAKERERE UNIVERSITY IN ORDER TO BUILD A RAIN FOCUS APP TO AID EFFICIENT PLANNING FOR A DAY.
}
\author{By\\Akol Sharon Norah, kyeswa Ivan Lutimba, Ntambi Isaac, Namulidwa Hellen}
\date{Mar 10, 2018}

\begin{document}

\maketitle

\newpage

\section{Introduction.}
Rain is liquid water in the form of droplets that have condensed from atmosphere and they become heavy enough to fall under gravity.
Water can be in the atmosphere, on land, in the ocean and even underground. It gets used over and over again through a water cycle. In this cycle water changes from liquid, solid, and gas (water vapour). Water vapour then gets into the atmosphere through a process called evaporation. This then turns water at the top of oceans, rivers and lakes into water vapour in the atmosphere using the energy from the sun. This vapour can also form snow and ice too. The water rises in the atmosphere and there it cools down and forms tiny water droplets through condensation. These then turn into clouds. When they combine together, they grow bigger and are too heavy to stay up in the air. This is when they will fall on the ground as rain, snow or hail by gravity.There are many factors that contribute to rainfall and they include the following:


\begin{itemize}
    \item Humidity: It is the amount of water vapour in the air. It rains more on the coasts than in an inland.

    \item Firebase Cloud Messaging- An instant messaging service that can quickly be implemented by a product
    \item Latitude:  It rains more in the areas near the equator than in the temperature zones and polar regions. The temperature is higher near the equator so there is more evaporation.
    \item Altitude: It rains more in higher areas than in low area because as air is forced over higher ground it cools, causing moist air to condense and fall as rainfall.
    \item Temperatures: At higher temperatures, the atmosphere may contain more water vapour thus increasing the chance of heavy rain showers.

    \end{itemize}





\end{document}
