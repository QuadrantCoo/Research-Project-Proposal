%%%%%%%%%%%%  Generated using docx2latex.com  %%%%%%%%%%%%%%

%%%%%%%%%%%%  v2.0.0-beta  %%%%%%%%%%%%%%

\documentclass[12pt]{article}
\usepackage{amsmath}
\usepackage{latexsym}
\usepackage{amsfonts}
\usepackage[normalem]{ulem}
\usepackage{array}
\usepackage{amssymb}
\usepackage{graphicx}

% If you are compiling in your own LaTeX editor and this 
 %part of code is throwing error then remove following few lines of code before \usepackage{subfig}
\usepackage[backend=biber,
style=numeric,
sorting=ynt
]{biblatex}\addbibresource{bibliography.bib}

\usepackage{subfig}
\usepackage{wrapfig}
\usepackage{wasysym}
\usepackage{enumitem}
\usepackage{adjustbox}
\usepackage{ragged2e}
\usepackage[svgnames,table]{xcolor}
\usepackage{tikz}
\usepackage{longtable}
\usepackage{changepage}
\usepackage{setspace}
\usepackage{hhline}
\usepackage{multicol}
\usepackage{tabto}
\usepackage{float}
\usepackage{multirow}
\usepackage{makecell}
\usepackage{fancyhdr}
\usepackage[toc,page]{appendix}
\usepackage[paperheight=11.69in,paperwidth=8.27in,left=1.0in,right=1.0in,top=1.0in,bottom=1.0in,headheight=1in]{geometry}
\usepackage[utf8]{inputenc}
\usepackage[T1]{fontenc}
\usepackage[hidelinks]{hyperref}
\usetikzlibrary{shapes.symbols,shapes.geometric,shadows,arrows.meta}
\tikzset{>={Latex[width=1.5mm,length=2mm]}}
\usepackage{flowchart}\TabPositions{0.5in,1.0in,1.5in,2.0in,2.5in,3.0in,3.5in,4.0in,4.5in,5.0in,5.5in,6.0in,}

\urlstyle{same}


 %%%%%%%%%%%%  Set Depths for Sections  %%%%%%%%%%%%%%

% 1) Section
% 1.1) SubSection
% 1.1.1) SubSubSection
% 1.1.1.1) Paragraph
% 1.1.1.1.1) Subparagraph


\setcounter{tocdepth}{5}
\setcounter{secnumdepth}{5}


 %%%%%%%%%%%%  Set Depths for Nested Lists created by \begin{enumerate}  %%%%%%%%%%%%%%


\setlistdepth{9}
\renewlist{enumerate}{enumerate}{9}
	\setlist[enumerate,1]{label=\arabic*)}
	\setlist[enumerate,2]{label=\alph*)}
	\setlist[enumerate,3]{label=(\roman*)}
	\setlist[enumerate,4]{label=(\arabic*)}
	\setlist[enumerate,5]{label=(\Alph*)}
	\setlist[enumerate,6]{label=(\Roman*)}
	\setlist[enumerate,7]{label=\arabic*}
	\setlist[enumerate,8]{label=\alph*}
	\setlist[enumerate,9]{label=\roman*}

\renewlist{itemize}{itemize}{9}
	\setlist[itemize]{label=$\cdot$}
	\setlist[itemize,1]{label=\textbullet}
	\setlist[itemize,2]{label=$\circ$}
	\setlist[itemize,3]{label=$\ast$}
	\setlist[itemize,4]{label=$\dagger$}
	\setlist[itemize,5]{label=$\triangleright$}
	\setlist[itemize,6]{label=$\bigstar$}
	\setlist[itemize,7]{label=$\blacklozenge$}
	\setlist[itemize,8]{label=$\prime$}

\setlength{\topsep}{0pt}\setlength{\parskip}{8.04pt}
\setlength{\parindent}{0pt}

 %%%%%%%%%%%%  This sets linespacing (verticle gap between Lines) Default=1 %%%%%%%%%%%%%%


\renewcommand{\arraystretch}{1.3}


%%%%%%%%%%%%%%%%%%%% Document code starts here %%%%%%%%%%%%%%%%%%%%



\begin{document}


%%%%%%%%%%%%%%%%%%%% Figure/Image No: 1 starts here %%%%%%%%%%%%%%%%%%%%

\begin{figure}[H]
	\begin{Center}
		\includegraphics[width=0.81in,height=0.7in]{./media/image1.jpeg}
	\end{Center}
\end{figure}


%%%%%%%%%%%%%%%%%%%% Figure/Image No: 1 Ends here %%%%%%%%%%%%%%%%%%%%

\setlength{\parskip}{6.0pt}
\par

\begin{Center}
{\fontsize{13pt}{15.6pt}\selectfont \textbf{MAKERERE\ \ \ \ \ \ \ \ \ \ \ \ \ \ \ \ \ \ \ \ \ \ \ \  UNIVERSITY}\par}
\end{Center}\par


\vspace{\baselineskip}
\setlength{\parskip}{0.0pt}
\begin{Center}
{\fontsize{13pt}{15.6pt}\selectfont \textbf{COLLEGE OF COMPUTING AND INFORMATION SCIENCES}\par}
\end{Center}\par

\begin{Center}
{\fontsize{13pt}{15.6pt}\selectfont \textbf{SCHOOL OF COMPUTING AND INFORMATICS TECHNOLOGY}\par}
\end{Center}\par

\begin{Center}
{\fontsize{13pt}{15.6pt}\selectfont \textbf{DEPARTMENT OF COMPUTER SCIENCE}\par}
\end{Center}\par

\begin{Center}
{\fontsize{13pt}{15.6pt}\selectfont \textbf{COURSE CODE: BIT 2207 }\par}
\end{Center}\par


\vspace{\baselineskip}
\begin{Center}
{\fontsize{13pt}{15.6pt}\selectfont \textbf{COURSE NAME}: RESEARCH METHODOLOGY\par}
\end{Center}\par

\begin{Center}
{\fontsize{13pt}{15.6pt}\selectfont \textbf{LECTURER}: MR. MWEBAZE ERNEST\par}
\end{Center}\par

\begin{Center}
{\fontsize{13pt}{15.6pt}\selectfont A RESEACH PROPOSAL \par}
\end{Center}\par


\vspace{\baselineskip}
 \tabto{0.35in}  \tabto{3.13in} {\fontsize{13pt}{15.6pt}\selectfont \textbf{BY:}\par}\par



%%%%%%%%%%%%%%%%%%%% Table No: 1 starts here %%%%%%%%%%%%%%%%%%%%


\begin{table}[H]
 			\centering
\begin{tabular}{p{2.16in}p{1.77in}p{1.57in}}
\hline
%row no:1
\multicolumn{1}{|p{2.16in}}{\Centering {\fontsize{13pt}{15.6pt}\selectfont \textbf{NAME}}} & 
\multicolumn{1}{|p{1.77in}}{\Centering {\fontsize{13pt}{15.6pt}\selectfont \textbf{REGISTRATION NO.}}} & 
\multicolumn{1}{|p{1.57in}|}{\Centering {\fontsize{13pt}{15.6pt}\selectfont \textbf{STUDENT NO.}}} \\
\hhline{---}
%row no:2
\multicolumn{1}{|p{2.16in}}{{\fontsize{13pt}{15.6pt}\selectfont NTAMBI ISAAC}} & 
\multicolumn{1}{|p{1.77in}}{\Centering {\fontsize{13pt}{15.6pt}\selectfont 16/U/10485/PS}} & 
\multicolumn{1}{|p{1.57in}|}{\Centering {\fontsize{13pt}{15.6pt}\selectfont 216014342}} \\
\hhline{---}
%row no:3
\multicolumn{1}{|p{2.16in}}{{\fontsize{13pt}{15.6pt}\selectfont AKOL SHARON NORAH}} & 
\multicolumn{1}{|p{1.77in}}{\Centering {\fontsize{13pt}{15.6pt}\selectfont 16/U/74}} & 
\multicolumn{1}{|p{1.57in}|}{\Centering {\fontsize{13pt}{15.6pt}\selectfont 216000979}} \\
\hhline{---}
%row no:4
\multicolumn{1}{|p{2.16in}}{{\fontsize{13pt}{15.6pt}\selectfont NAMULINDA HELLEN}} & 
\multicolumn{1}{|p{1.77in}}{\Centering {\fontsize{13pt}{15.6pt}\selectfont 16/U/900}} & 
\multicolumn{1}{|p{1.57in}|}{\Centering {\fontsize{13pt}{15.6pt}\selectfont 216000850}} \\
\hhline{---}
%row no:5
\multicolumn{1}{|p{2.16in}}{{\fontsize{13pt}{15.6pt}\selectfont KYESWA LUTIMBA IVAN}} & 
\multicolumn{1}{|p{1.77in}}{\Centering {\fontsize{13pt}{15.6pt}\selectfont 16/U/512}} & 
\multicolumn{1}{|p{1.57in}|}{\Centering {\fontsize{13pt}{15.6pt}\selectfont 216001516}} \\
\hhline{---}

\end{tabular}
 \end{table}


%%%%%%%%%%%%%%%%%%%% Table No: 1 ends here %%%%%%%%%%%%%%%%%%%%


\vspace{\baselineskip}

\vspace{\baselineskip}

\vspace{\baselineskip}

\vspace{\baselineskip}

\vspace{\baselineskip}

\vspace{\baselineskip}

\vspace{\baselineskip}

\vspace{\baselineskip}

\vspace{\baselineskip}

\vspace{\baselineskip}

\vspace{\baselineskip}

\vspace{\baselineskip}
\begin{Center}
{\fontsize{13pt}{15.6pt}\selectfont \textbf{ESTIMATING TO A GOOD APPROXIMATION THE LIKELYHOOD OF RAIN IN MAKERERE UNIVERSITY IN ORDER TO BUILD A RAIN FOCUS APP TO AID EFFICIENT PLANNING FOR A DAY.}\par}
\end{Center}\par


\vspace{\baselineskip}
\begin{Center}
{\fontsize{13pt}{15.6pt}\selectfont \textbf{INTRODUCTION.}\par}
\end{Center}\par

{\fontsize{13pt}{15.6pt}\selectfont Rain is liquid water in the form of droplets that have condensed from atmosphere and the become heavy enough to fall under gravity.\par}\par


\vspace{\baselineskip}
{\fontsize{13pt}{15.6pt}\selectfont \textbf{How rain is formed.}\par}\par

{\fontsize{13pt}{15.6pt}\selectfont  Water can be in the atmosphere, on land, in the ocean and even underground. It gets used over and over again through a water cycle. In this cycle water changes from liquid, solid, and gas (water vapour). Water vapour then gets into the atmosphere through a process called evaporation. This then turns water at the top of oceans, rivers and lakes into water vapour in the atmosphere using the energy from the sun. This vapour can also form snow and ice too. The water rises in the atmosphere and there it cools down and forms tiny water droplets through condensation. These then turn into clouds. When they combine together, they grow bigger and are too heavy to stay up in the air. This is when they will fall on the ground as rain, snow or hail by gravity.\par}\par

{\fontsize{13pt}{15.6pt}\selectfont There are many factors that contribute to rainfall and they include the following.\par}\par

{\fontsize{13pt}{15.6pt}\selectfont \textbf{Humidity:} It is the amount of water vapour in the air. It rains more on the coasts than in an inland.\par}\par

{\fontsize{13pt}{15.6pt}\selectfont \textbf{Latitude:}\  It rains more in the areas near the equator than in the temperature zones and polar regions. The temperature is higher near the equator so there is more evaporation.\par}\par

{\fontsize{13pt}{15.6pt}\selectfont \textbf{Altitude:} It rains more in higher areas than in low area because as air is forced over higher ground it cools, causing moist air to condense and fall as rainfall.\par}\par

{\fontsize{13pt}{15.6pt}\selectfont \textbf{Temperatures:} At higher temperatures, the atmosphere may contain more water vapour thus increasing the chance of heavy rain showers.\par}\par


\vspace{\baselineskip}
\begin{Center}
{\fontsize{13pt}{15.6pt}\selectfont \textbf{PROBLEM STATEMENT}\par}
\end{Center}\par

{\fontsize{13pt}{15.6pt}\selectfont There is a problem of poor planning for a day with students in Makerere University during the rainy seasons. According to us as group, we seek to help students know the likelihood of rain and this will help them to efficiently plan for their days most especially in the rainy seasons.\par}\par

{\fontsize{13pt}{15.6pt}\selectfont Despite\ the fact that there is a weather station in the University, the students do not have access to the analytical results of the information collected and therefore the station does not benefit them. Though a few may act by carrying umbrellas for assurance but end up regretting why they did so if it does not rain.  \par}\par

{\fontsize{13pt}{15.6pt}\selectfont In response to this problem, our study proposes to investigate likelihood of rain in Makerere University using a Rain Focus App will help resolve the situation. This will give students prior knowledge of the likelihood of it raining the next day hence enhancing efficient planning for the day.\par}\par


\vspace{\baselineskip}
\begin{Center}
{\fontsize{13pt}{15.6pt}\selectfont \textbf{OBJECTIVES}\par}
\end{Center}\par


\vspace{\baselineskip}
{\fontsize{13pt}{15.6pt}\selectfont \textbf{Main objective:}\par}\par

\begin{itemize}
	\item {\fontsize{13pt}{15.6pt}\selectfont To develop a web application that broadcasts Makerere University rain focus.\par}\par


\vspace{\baselineskip}
{\fontsize{13pt}{15.6pt}\selectfont \textbf{Specific objectives:}\par}\par

	\item {\fontsize{13pt}{15.6pt}\selectfont To find out if there is any server that freely broadcasts weather data for areas around Makerere university.\par}\par

	\item {\fontsize{13pt}{15.6pt}\selectfont To find a way to access the weather data.\par}\par

	\item {\fontsize{13pt}{15.6pt}\selectfont  To analyse the weather data to understand what it means.\par}\par

	\item {\fontsize{13pt}{15.6pt}\selectfont To design the web application and represent the weather data plus supportive information for it.\par}\par

	\item {\fontsize{13pt}{15.6pt}\selectfont To find a reliable web server for hosting the web application.\par}
\end{itemize}\par


\vspace{\baselineskip}
\begin{Center}
{\fontsize{13pt}{15.6pt}\selectfont \textbf{LITERATURE REVIEW}\par}
\end{Center}\par


\vspace{\baselineskip}
{\fontsize{13pt}{15.6pt}\selectfont \textbf{How weather is determined:}\par}\par

{\fontsize{13pt}{15.6pt}\selectfont \textcolor[HTML]{333333}{Weather forecasts are made by collecting as much data as possible about the current state of the atmosphere (particularly the temperature, humidity and wind) and using understanding of atmospheric processes (through meteorology \textsuperscript{[1]}) to determine how the atmosphere evolves in the future.}\par}\par

\setlength{\parskip}{7.44pt}
{\fontsize{13pt}{15.6pt}\selectfont \textcolor[HTML]{333333}{However, the chaotic nature of the atmosphere and incomplete understanding of the processes mean that forecasts become less accurate as the range of the forecast increases.}\par}\par

{\fontsize{13pt}{15.6pt}\selectfont \textcolor[HTML]{333333}{During the data assimilation process, information gained from the observations is used in conjunction with a numerical model's most recent forecast for the time that observations were made to produce the meteorological analysis. These weather prediction models are computer simulations of the atmosphere.}\par}\par

{\fontsize{13pt}{15.6pt}\selectfont \textcolor[HTML]{333333}{They take the analysis as the starting point and evolve the state of the atmosphere forward in time using understanding of physics and fluid dynamics \textsuperscript{[2][3]}.}\par}\par


\vspace{\baselineskip}

\vspace{\baselineskip}
{\fontsize{13pt}{15.6pt}\selectfont \textbf{\textcolor[HTML]{222222}{Weather forecasting:}}\par}\par

{\fontsize{13pt}{15.6pt}\selectfont  Is the application of science and technology to predict the conditions of the \href{https://en.wikipedia.org/wiki/Earth$\%$ 27s\_atmosphere}{atmosphere} for a given location and time. Human beings have attempted to predict the \href{https://en.wikipedia.org/wiki/Weather}{weather} informally for \href{https://en.wikipedia.org/wiki/Millennia}{millennia} and formally since the 19th century. Weather forecasts are made by collecting quantitative \href{https://en.wikipedia.org/wiki/Data}{data} about the current state of the atmosphere at a given place and using \href{https://en.wikipedia.org/wiki/Meteorology}{meteorology} to project how the atmosphere will change \textsuperscript{[3]}.\par}\par


\vspace{\baselineskip}
{\fontsize{13pt}{15.6pt}\selectfont \textbf{How weather information is broadcasted.}\par}\par

\setlength{\parskip}{4.56pt}
\section*{According to a journal written by Walt Hickey, after a survey on $``$Where People Go To Check The Weather$"$ , analysing the survey results, we can make conclusions that people seek out weather information from quick and easily accessible sources [4]. }
\addcontentsline{toc}{section}{According to a journal written by Walt Hickey, after a survey on $``$Where People Go To Check The Weather$"$ , analysing the survey results, we can make conclusions that people seek out weather information from quick and easily accessible sources [4]. }

\vspace{\baselineskip}
\section*{Conclusion}
\addcontentsline{toc}{section}{Conclusion}
\section*{Weather focus has evolved over a long time a variety of models and technologies have been developed to address its argent demand and people need this information in quick, clear and easily accessible means.}
\addcontentsline{toc}{section}{Weather focus has evolved over a long time a variety of models and technologies have been developed to address its argent demand and people need this information in quick, clear and easily accessible means.}

\vspace{\baselineskip}
\setlength{\parskip}{7.44pt}
{\fontsize{13pt}{15.6pt}\selectfont \textbf{METHODOLOGY}\par}\par

{\fontsize{13pt}{15.6pt}\selectfont There are several servers with reliable weather data which have free plans and associated API’s providing data in JSON, XML, or HTML format. We are going to obtain weather information for Makerere region and design it for display using simple high quality design styling and layout specifically for Makerere. We would also love to select a web hosting server that is closest to the area to reduce on the network delay period so as to improve performance by hosting the application there. \par}\par


\vspace{\baselineskip}


%%%%%%%%%%%%%%%%%%%% Table No: 2 starts here %%%%%%%%%%%%%%%%%%%%


\begin{table}[H]
 			\centering
\begin{tabular}{p{0.29in}p{2.26in}p{3.11in}}
\hline
%row no:1
\multicolumn{1}{|p{0.29in}}{{\fontsize{13pt}{15.6pt}\selectfont \textbf{No}}} & 
\multicolumn{1}{|p{2.26in}}{{\fontsize{13pt}{15.6pt}\selectfont \textbf{Objective}}} & 
\multicolumn{1}{|p{3.11in}|}{{\fontsize{13pt}{15.6pt}\selectfont \textbf{Method/Technique}}} \\
\hhline{---}
%row no:2
\multicolumn{1}{|p{0.29in}}{{\fontsize{13pt}{15.6pt}\selectfont \textbf{1}}} & 
\multicolumn{1}{|p{2.26in}}{{\fontsize{13pt}{15.6pt}\selectfont Find a server with weather data }} & 
\multicolumn{1}{|p{3.11in}|}{{\fontsize{13pt}{15.6pt}\selectfont Download weather API or obtain API Key }} \\
\hhline{---}
%row no:3
\multicolumn{1}{|p{0.29in}}{{\fontsize{13pt}{15.6pt}\selectfont \textbf{2}}} & 
\multicolumn{1}{|p{2.26in}}{{\fontsize{13pt}{15.6pt}\selectfont Access the data}} & 
\multicolumn{1}{|p{3.11in}|}{{\fontsize{13pt}{15.6pt}\selectfont Querying server through weather API over internet protocols using script queries.}} \\
\hhline{---}
%row no:4
\multicolumn{1}{|p{0.29in}}{{\fontsize{13pt}{15.6pt}\selectfont \textbf{3}}} & 
\multicolumn{1}{|p{2.26in}}{{\fontsize{13pt}{15.6pt}\selectfont Analyse weather data}} & 
\multicolumn{1}{|p{3.11in}|}{{\fontsize{13pt}{15.6pt}\selectfont Statistical compilation and content analysis}} \\
\hhline{---}
%row no:5
\multicolumn{1}{|p{0.29in}}{{\fontsize{13pt}{15.6pt}\selectfont \textbf{4}}} & 
\multicolumn{1}{|p{2.26in}}{{\fontsize{13pt}{15.6pt}\selectfont Design the web application}} & 
\multicolumn{1}{|p{3.11in}|}{{\fontsize{13pt}{15.6pt}\selectfont Script with HTML,CSS,JS using Sublime Text editor and Chrome browser}} \\
\hhline{---}
%row no:6
\multicolumn{1}{|p{0.29in}}{{\fontsize{13pt}{15.6pt}\selectfont \textbf{5}}} & 
\multicolumn{1}{|p{2.26in}}{{\fontsize{13pt}{15.6pt}\selectfont  Hosting the application}} & 
\multicolumn{1}{|p{3.11in}|}{{\fontsize{13pt}{15.6pt}\selectfont Register domain name, host content and configure DNS using Gmail.}} \\
\hhline{---}

\end{tabular}
 \end{table}


%%%%%%%%%%%%%%%%%%%% Table No: 2 ends here %%%%%%%%%%%%%%%%%%%%


\vspace{\baselineskip}

\vspace{\baselineskip}
{\fontsize{13pt}{15.6pt}\selectfont \textbf{Design results}\par}\par


\vspace{\baselineskip}
{\fontsize{13pt}{15.6pt}\selectfont \textbf{\textcolor[HTML]{333333}{REFERRENCES}}\par}\par

\begin{enumerate}
	\item {\fontsize{13pt}{15.6pt}\selectfont \textit{\textcolor[HTML]{333333}{$ \string^ $  Wikipedia - \href{https://en.wikipedia.org/wiki/Meteorology}{https://en.wikipedia.org/wiki/Meteorology}}}\par}\par

	\item {\fontsize{13pt}{15.6pt}\selectfont \textit{\textcolor[HTML]{333333}{$ \string^ $  \href{https://www.sciencedaily.com/terms/weather\_forecasting.htm}{https://www.sciencedaily.com/terms/weather\_forecasting.htm}}}\par}\par

	\item {\fontsize{13pt}{15.6pt}\selectfont \textit{\textcolor[HTML]{333333}{$ \string^ $  Wikipedia - \href{https://en.wikipedia.org/wiki/Weather\_forecasting}{https://en.wikipedia.org/wiki/Weather\_forecasting}}}\par}\par

	\item \href{https://fivethirtyeight.com/contributors/walt-hickey/}{$ \string^ $  Walt Hickey}{\fontsize{13pt}{15.6pt}\selectfont \textit{ - \href{https://fivethirtyeight.com/features/weather-forecast-news-app-habits/}{https://fivethirtyeight.com/features/weather-forecast-news-app-habits/}}\par}
\end{enumerate}\par


\vspace{\baselineskip}

\vspace{\baselineskip}

\vspace{\baselineskip}

\vspace{\baselineskip}

\vspace{\baselineskip}

\vspace{\baselineskip}

\vspace{\baselineskip}
\end{document}